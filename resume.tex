\documentclass{res}
\usepackage[colorlinks=true,urlcolor=blue]{hyperref}
\setlength{\textheight}{9.5in} % increase text height to fit on 1-page

\begin{document}

\name{Youngung Jeong, PhD\\[12pt]}     % the \\[12pt] adds a blank

\address{
  \bf Department of Materials Science and Engineering\\
  \bf Changwon National University\\
  \bf 20 Changwondaehak-ro\\
  \bf Changwon, Gyeongnam, Republic of Korea\\
  \bf zip: 51140\\
  \bf +82-055-213-3694\\
  \bf yjeong@changwon.ac.kr\\
}

% line after name

% \address{\bf Center for Automotive Lightweighting\\
%   \bf National Institute of Standards and Technology\\
%   \bf US Department of Commerce\\
%   \bf 100 Bureau Drive Stop 8553,\\
%   \bf Gaithersburg, MD, US\\
%   \bf zip: 20899-8553
%   \bf +1 (301) 975-5028\\
%   \bf youngung.jeong@nist.gov\\
%   \bf youngung.jeong@gmail.com}
% \address{\bf International Center for Automotive Research\\
%   Clemson University\\
%   4 Research Drive, Greenville, SC 29607\\
%   youngung.jeong@gmail.com\\
%   younguj@clemson.edu\\


\begin{resume}
  \section{EDUCATION}
  \begin{itemize}
  \item PhD, Graduate Institute of Ferrous Technology, Pohang University of Science and Technology, Pohang, South Korea, March 2014\\
    Advisor: Professor Fr\'ed\'eric Barlat, Co-advisor: Professor Myoung-Gyu Lee\\
    Thesis title: {\it Application of self-consistent crystal plasticity framework as a constitutive description for commercial steel sheets}
  \item MS, Graduate Institute of Ferrous Technology, Pohang University of Science and Technology, Pohang, South Korea, March 2010\\
    Advisor: Professor Fr\'ed\'eric Barlat
  \item BS, Materials Science and Engineering, Hanyang University, Seoul, South Korea, Feb 2008
  \end{itemize}

  % \section{PERSONAL}
  % Born on March 15, 1982
  % Citizenship: Republic of Korea
  \section{EXPERIENCES}
  \begin{itemize}
  \item Assistant professor, Department of Materials Science and Engineering, Changwon National University, South Korea, 2017 March - Present
  \item Post Doctorate Researcher, Graduate Institute of Ferrous Technology, POSTECH, South Korea, 2016 Dec - 2017 Feb.
  \item Research Scientist (Postdoc), International Center for Automotive Research, Clemson University, South Carolina, US, 2016 March - 2016 Nov.
  \item Post Doctorate Researcher (PREP Fellowship through University of Maryland College Park), Center for Automotive Lightweighting,
    National Institute of Standards and Technology, US Department of Commerce, USA, 2014 March - 2016 Feb\\
    - Conducted a series of experiments to obtain multiaxial constitutive behavior of an interstitial-free steel\\
    - Measured multiaxial flow stress using X-ray diffraction for metal sheets subjected to various multiaxial loading conditions (DiffStress package)\\
    - Performed the strain analysis using digital image correlation technique to determine the forming limit diagram of the IF steel\\
    - Developed the VPSC-FLD model to predict forming limit diagram of engineering metal sheets (VPSC-FLD package)\\
    - Developed VPSC-based model to link with continuum-scale phenomenological model (VPSC-RGVB-YLD forked from VPSC-FLD)
  \item Research Affiliate, Materials Science Technology Division 8, Los Alamos National Laboratory, Los Alamos, New Mexico, US, 2012 April - September \\
    - Leading role in implementing a phase transformation kinetics model into Elasto-ViscoPlastic Self-Consistent crystal plasticity model
  \item Guest Researcher, Materials Science and Engineering Division, National Institute of Standards and Technology, US Department of Commerce, Gaithersburg, MD, USA, 2011\\
    - Conducted experiments to obtain multiaxial stress-strain measurements using digital image correlation and in-situ X-ray technique
  \end{itemize}

  \section{RESEARCH INTERESTS}
  \begin{itemize}
  \item Computational modeling and simulations of polycrystalline materials
  \item Martensitic phase transformation, Transformation-Induced Plasticity (TRIP) steels
  \item Analysis of crystallographic texture development and its evolution induced by plastic deformation
  \item Stress analysis on the basis of X-ray diffraction method for elastically anisotropic polycrystalline metals
  \item Forming limit prediction for metal sheets
  \item Constitutive modeling of advanced high strength steels
  \item Digital Image Correlation technique for strain analysis
  \item Linking crystal plasticity framework with continuum-scale phenomenological model
  \item Optimization of crystallographic texture to improve mechanical performance of metal sheets
  \end{itemize}

  \section{TECHNICAL SKILLS}
  \subsection{Experimental skills}
  \begin{itemize}
  \item Uniaxial tension, shear, hydraulic bulge test
  \item Biaxial tests using cruciform piece and Marciniak
  \item Forming limit analysis using digital image correlation
  \item X-ray and neutron diffraction for residual stress, crystallographic texture, and phase fraction measurements
  \end{itemize}

  \subsection{Computer skills}
  \begin{itemize}
  \item Programming: FORTRAN, C, $C^{++}$, Python (including various scientific libraries such as SciPy, NumPy, and Matplotlib), Bash script
  \item Advanced data-driven analyses using IPython and notebook.
  \item Parallel computation using Python's multiprocessing package, Linux Cluster
  \item OS: Linux, Windows, MacOS
  \item OA: MS-Office (Excel, MS-Word, Power Point), Emacs, \LaTeX
  \item Technical programs: Abaqus (UMAT development), VIC3D (DIC software)
  \item Version control: Git
  \item GitHub account: https://www.github.com/youngung
  \end{itemize}

  \section{TECHNICAL COMPUTER PROGRAMS DEVELOPMENT}
  Public repositories are accessible through my GitHub Account: \url{https://github.com/youngung}
  \begin{itemize}
  \item \href{https://github.com/youngung/abaquspy}{AbaqusPy}: My personal package including UMAT for elasto-plastic constitutive model, which can be used for sheet metal forming/springback prediction using Abaqus (under development)
  \item VPSC-FLD: a numerical computer program to predict forming limit diagram with parallel computations using the viscoplastic self-consistent crystal plasticity framework (Hosted privately in the USNISTGOV account in GitHub)
  \item VPSC-YLD: a python package that runs a series of VPSC simulations in parallel in order to characterize advanced anisotropic yield functions
  \item \href{https://github.com/usnistgov/DiffStress}{DiffStress}: A Python package to estimate the residual stress based on interplanar \emph{d}-spacings measured by X-ray diffraction technique for fully anisotropic (textured) materials subjected to multiaxial loadings (Hosted in the USNISTGOV account in GitHub)
  \item \href{https://github.com/youngung/texture}{TEXT}: a set of Python scripts to post-process orientation distribution functions provided in a discrete data\\
    - Any crystal structure, cubic/hexagonal\\
    - Graphical tool to plot pole figures using matplotlib package\\
    - Publicly Hosted in personal account in GitHub
  \item \href{https://github.com/youngung/mk}{MK}: Marciniak-Kuczynski model using macro-mechanical constitutive descriptions (currently under development)
  \end{itemize}

  \section{Misc.}
  \begin{itemize}
  \item Manual for VPSC-FLD, \underline{Youngung Jeong}, 2014 (user manual for VPSC-FLD program)
  \end{itemize}

  \section{LANGUAGES}
  English (Fluent), Korean (Mother Tongue)

  \section{JOURNAL ARTICLES}
  \begin{itemize}
  \item \underline{Y. Jeong}$^*$, T. Gn\"{a}upel-Herold, M. Iadicola, A. Creuziger, \href{https://doi.org/10.1107/S1600576716013662}{Uncertainty in flow stress measurements using X-ray diffraction for sheet metals subjected to large plastic deformations}, Journal of Applied Crystallography 49, p1991-2004, 2016
  \item D. Steglich, \underline{Y. Jeong}$^*$, \href{http://dx.doi.org/10.1016/j.ijmecsci.2016.08.013}{Texture-based forming limit prediction for Mg sheet alloys ZE10 and AZ31}, International Journal of Mechanical Sciences 117, p102-114, 2016
  \item \underline{Y. Jeong}$^*$, F. Barlat, C. Tom\'{e}, W. Wen, \href{http://dx.doi.org/10.1016/j.ijplas.2016.07.015}{A comparative study between micro- and macro-mechanical constitutive models developed for complex loading scenarios}, International Journal of Plasticity, 2016 (In press)
  \item \underline{Y. Jeong}$^*$, M.-S. Pham, M. Iadicola, A. Creuziger, T. Foecke, \href{http://dx.doi.org/10.1088/0965-0393/24/5/055005}{Forming limit prediction using a self-consistent crystal plasticity framework: a case study for BCC fiber textures}, Modelling and Simulation in Materials Science and Engineering 24(5), 055002 (21 pp), 2016
  \item \underline{Y. Jeong}$^*$, M. Iadicola, T. Gn\"{a}upel-Herold, A. Creuziger, \href{http://dx.doi.org/10.1016/j.actamat.2016.04.013}{Multiaxial constitutive behavior of an interstitial-free steel: measurements through X-ray and digital image correlation}, Acta Materialia 112, 84-93, 2016
  \item H. Wang$^*$, \underline{Y. Jeong}, B. Clausen, Y. Liu, R. J. McCabe, F. Barlat, C. N. Tom\'{e}, \href{http://dx.doi.org/10.1016/j.msea.2015.09.108}{Effect of martensitic phase transformation on the behavior of 304 austenitic stainless steel under tension}, Materials Science and Engineering A {\bf 649}, 174-183, 2016
  \item \underline{Y. Jeong}, T. Gn\"{a}upel-Herold, F. Barlat, M. Iadicola, A. Creuziger, M.-G. Lee$^*$, \href{http://dx.doi.org/10.1016/j.ijplas.2014.06.009}{Evaluation of biaxial flow stress based on Elasto-Viscoplastic Self-Consistent analysis of X-ray Diffraction Measurements}, International Journal of Plasticity {\bf 66}, 103-118, 2015
  \item \underline{Y. Jeong}$^*$, F. Barlat, M.-G. Lee,  \href{http://dx.doi.org/10.1088/0965-0393/20/2/024009}{Application of crystal plasticity to an austenitic stainless steel}, Modelling and Simulation in Materials Science and Engineering {\bf 20}, 024009, 2012
  \item D. Steglich$^*$, \underline{Y. Jeong}, M. O. Andar, T. Kuwabara, \href{http://dx.doi.org/10.1016/j.ijsolstr.2012.06.017}{Biaxial Deformation Behavior of AZ31 Magnesium Alloy: Crystal-Plasticity-Based Prediction and Experimental Validation}, International Journal of Solids and Structure {\bf 49} (25), 3551-3561, 2012
  \end{itemize}

  % \section{JOURNAL PAPERS in preparation}
  % \begin{itemize}
  % \end{itemize}

  \section{CONFERENCE PROCEEDINGS}
  \begin{itemize}
  \item F. Barlat, \underline{Y. Jeong}, J. Ha, C Tom\'{e}, Myoung-Gyu Lee, W. Wen, Advances in Constitutive Modeling of Plasticity for Forming Applications, (Submitted) AEPA2016, 2016
  \item \underline{Y. Jeong}, F. Barlat, C. Tom\'{e}, W. Wen, \href{http://dx.doi.org/10.1063/1.4963544}{Validation of Homogeneous Anisotropic Hardening Approach Based on Crystal Plasticity}, ESAFORM, 2016
  \item \underline{Y. Jeong}, M.-S. Pham, M. Iadicola, A. Creuziger, \href{https://doi.org/10.4028/www.scientific.net/KEM.651-653.193}{Forming limit predictions using a self-consistent crystal plasticity model: a parametric study}, Key Engineering Materials {\bf 651}, 193-198, 2015
  \item \underline{Y. Jeong}, F. Barlat, M.-G. Lee, \href{https://doi.org/10.4028/www.scientific.net/MSF.702-703.224}{Microstructural and crystallographic aspects of yield surface evolution}, Materials Science Forum {\bf 702}, 224-228, 2011
  \item \underline{Y. Jeong}, F. Barlat, M.-G. Lee, \href{https://doi.org/10.4028/www.scientific.net/MSF.702-703.204}{Crystal Plasticity Predictions of Forward-Reverse Simple Shear Flow Stress}, Materials Science Forum {\bf 1517} (702), 204-207, 2011
  \end{itemize}

  \section{INTERNATIONAL CONFERENCES}
  \begin{itemize}
  \item F. Barlat, \underline{Y. Jeong}, C. Tom\'{e}, {\it Validation of Homogeneous Anisotropic Hardening Approach Based on Crystal Plasticity},  Plasticity Conference 2016
  \item \underline{Y. Jeong}, M.-S. Pham, M. Iadicola, A. Creuziger, {\it VPSC-FLD: forming limit predictions for BCC fibers},  TMS 2015, Orlando
  \item \underline{Y. Jeong}, M.-S. Pham, M. Iadicola, A. Creuziger, {\it Forming limit diagram predictions using a self-consistent crystal plasticity model: a parametric study} , ESAFORM 2015
  \item \underline{Y. Jeong}, C. Tom\'{e}, B. Clausen, H. Wang, M.-G. Lee, F. Barlat, {\it A mean field polycrystal plasticity framework for strain-induced martensitic transformation}, Symposium of Plasticity 2014
  \item C. Tom\'{e}, \underline{Y. Jeong}, H. Wang, B. Clausen, F. Barlat, {\it In-situ Neutron  Measurement and Modeling of Martensitic Phase Transformation} International Conference on Multiscale Materials Modeling, 2014
  \item \underline{Y. Jeong}, M.-S. Pham, M. Iadicola, A. Creuziger, {\it Evaluation of an X-ray stress technique by means of crystal plasticity},  Materials Science \& Technology 2014 Fall meeting
  \item D. Steglich, \underline{Y. Jeong}, M. O. Andar, T. Kuwabara, {\it Biaxial Deformation Behaviour of AZ31 Magnesium Alloy: Crystal-Plasticity-Based Prediction and Experimental Validation}, NUMISHEET, 2011
  \item \underline{Y. Jeong}, F. Barlat, {\it Plastic Anisotropy in Multiphase Steels}, TMS Annual Meeting \& Exhibition, 2011
  \item \underline{Y. Jeong}, F. Barlat, M.-G. Lee, {\it Crystal plasticity predictions of cyclic simple shear flow stress}, International Conference on the Textures of Materials 16, 2011
  \item \underline{Y. Jeong}, F. Barlat, M.-G. Lee, {\it Microstructural and crystallographic aspects of yield surface evolution}, International Conference on the Textures of Materials 16, 2011
  \end{itemize}

  \section{DOMESTIC CONFERENCES}
  \begin{itemize}
  \item \underline{Y. Jeong}, M.-S. Pham, M. Iadicola, A. Creuziger, Experimental validation for VPSC-FLD and its application for various BCC fiber textures, NADDRG (North American Deep Drawing Research Group) 2015, Evanston IL, USA
  \item \underline{Y. Jeong}, D. Steglich, M.-G. Lee, F. Barlat, Investigation on the exponent of non-quadratic anisotropic yield surface models by crystal plasticity, KSTP 2010 fall meeting, Jeju, Korea
  \end{itemize}

  \section{WORKSHOPS}
  \begin{itemize}
  \item Young Researcher's Asian Workshop on Advanced Forming Technology, Nagoya, Japan, The Iron and Steel Institute of Japan, 2010
  \end{itemize}

  \section{SERVICES}
  \begin{itemize}
  \item Co-organized a symposium of {\it Material Behavior Characterization via Multi-Directional Deformation of Sheet Metal} in TMS, 2015
  \item Peer-review service: JOM, International Journal of Material Forming
  \end{itemize}

  \section{REFERENCES}
  \begin{itemize}
  \item Professor Frederic Barlat, GIFT, POSTECH (f.barlat@postech.ac.kr)
  \item Professor Myoung-Gyu Lee, Korea University (myounglee@korea.ac.kr)
  \item Drs. Carlos Tome and Ricardo Lebensohn (tome@lanl.gov, lebenso@lanl.gov)
  \item Drs. Timothy Foecke, Mark Iadicola, and Adam Creuziger at NIST Center for Automotive Lightweighting (timothy.foecke@nist.gov, mark.iadicola@nist.gov, adam.creuziger@nist.gov)
  \item Dr. Thomas Gn\"aupel-Herold, NIST Center for Neutron Research (thomas.gnaeupel-herold@nist.gov)
  \item Professor Fadi Abu-Farha, Clemson University (fadi@clemson.edu)
  \end{itemize}

\end{resume}
\end{document}
