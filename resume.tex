\documentclass{res}
\setlength{\textheight}{9.5in} % increase text height to fit on 1-page

\begin{document}

\name{Youngung Jeong\\[12pt]}     % the \\[12pt] adds a blank
% line after name

\address{\bf Center for Automotive Lightweighting \\
  \bf National Institute of Standards and Technology \\
  \bf US Department of Commerce \\
  \bf 100 Bureau Drive Stop 8553, \\
  \bf Gaithersburg, MD, US \\
  \bf zip: 20899-8553
  \bf +1 (301) 975-5028,  youngung.jeong@nist.gov}

\begin{resume}
  \section{EDUCATION}
  \begin{itemize}
  \item Bachelor, Materials Science and Engineering, Hanyang University, Seoul, Korea, Feb 2008
  \item PhD, Advisor: Professor Fr\'ed\'eric Barlat, Co-advisor: Professor Myoung-Gyu Lee, Graduate Institute of Ferrous Technology, Pohang University of Science and Technology, Korea, March 2010
  \end{itemize}

  % \section{PERSONAL}
  % Born on March 15, 1982
  % Citizenship: Republic of Korea
  \section{EXPERIENCES}
  \begin{itemize}
  \item Guest Researcher, Metallurgy Division, National Institute of
    Standards and Technology, US Department of Commerce, Gaithersburg,
    MD, USA, 2011\\
    - Conducted experiments to obtain multiaxial stress-strain measurements
    using digital image correlation and in-situ X-ray technique
  \item Research Affiliate, Materials Science Technology Division 8, Los Alamos
    National Laboratory, Los Alamos, New Mexico, US, 2012 April - September \\
    - Leading role in implementing a phase transformation kinetics model into Elasto-ViscoPlastic Self-Consistent crystal plasticity model
  \item Post Doctorate Researcher (PREP Fellowship through University of Maryland College Park), Center for Automotive Lightweighting,
    National Institute of Standards and Technology, US Department of Commerce, USA, 2014 March - present\\
    - Conducted a series of experiments to obtain multiaxial constitutive behavior of an interstitial-free steel\\
    - Measured multiaxial flow stress using X-ray diffraction for metal sheets subjected to various multiaxial loading conditions (RS package)\\
    - Performed the strain analysis using digital image correlation technique to determine the forming limit diagram of the IF steel\\
    - Developed the VPSC-FLD model to predict forming limit diagram of engineering metal sheets (VPSC-FLD)\\
    - Developing VPSC-based model to link with continuum-scale phenomenological model
  \end{itemize}

  \section{RESEARCH INTERESTS}
  \begin{itemize}
  \item Computational modeling and simulations for polycrystal materials
  \item Martensitic phase transformation
  \item Crystallographic texture development induced by plastic deformation
  \item Stress analysis on the basis of X-ray diffraction method for elastically anisotropic polycrystalline metals
  \item Forming limit prediction for metal sheets
  \item Constitutive modeling of advanced high strength steels
  \item Digital Image Correlation for strain analysis
  \item Linking crystal plasticity framework with continuum-scale phenomenological model
  \end{itemize}

  \section{COMPUTER SKILLS}
  \begin{itemize}
  \item Programming: FORTRAN, C, $C^{++}$, and Python
  \item Parallel computation
  \item OS: Multi years (+5) of experience with Linux and MS Windows
  \item OA: MS-Office (Excel, MS-Word, Power Point), Emacs, \LaTeX
  \item Technical programs: Abaqus, VIC3D (DIC software), NumPy, SciPy, Matplotlib, IPython
  \item Version control: Git
  \item GitHub account: https://github.com/youngung
  \end{itemize}

  \section{TECHNICAL COMPUTER PROGRAMS DEVELOPMENT}
  \begin{itemize}
  \item VPSC-FLD: a numerical computer program to predict forming limit diagram with parallel computations using the viscoplastic self-consistent crystal plasticity framework.\\
    - Hosted in the USNISTGOV account in GitHub
  \item RS: A Python package to estimate the residual stress based on interplanar \emph{d}-spacings measured by X-ray diffraction technique for fully anisotropic (textured) materials subjected to multiaxial loadings\\
    - Hosted in the USNISTGOV account in GitHub
  \item VPSC-YLD: a python package that runs a series of VPSC simulations in parallel in order to characterize advanced anisotropic yield functions
  \item TEXT: a set of Python scripts to post-process orientation distribution functions provided in a discrete data
    - Any crystal structure, cubic/hexagonal
    - Graphical tool to plot pole figures using matplotlib package.
    - Publicly Hosted in personal account in GitHub
  \end{itemize}

  \section{Misc.}
  \begin{itemize}
  \item Manual for VPSC-FLD, \underline{Youngung Jeong}, 2014 (user manual for VPSC-FLD program)
  \end{itemize}

  \section{LANGUAGES}
  English (Fluent), Korean (Mother Tongue)

  \section{JOURNAL PAPERS}
  \begin{itemize}
  \item 2015 Evaluation of biaxial flow stress based on Elasto-Viscoplastic Self-Consistent analysis of X-ray Diffraction Measurements, International Journal of Plasticity, \underline{Y. Jeong}, T. Gn\"{a}upel-Herold, F. Barlat, M. Iadicola, A. Creuziger, M.-G. Lee
  \item 2012 Application of crystal plasticity to an austenitic stainless steel, MSMSE, \underline{Y. Jeong}, F. Barlat, M.-g. Lee,
  \item 2012 Biaxial Deformation Behavior of AZ31 Magnesium Alloy: Crystal-Plasticity-Based Prediction and Experimental Validation, IJSS, D. Steglich, \underline{Y. Jeong}, M. O. Andar, T. Kuwabara
  \end{itemize}

  \section{JOURNAL PAPERS in preparation}
  \begin{itemize}
  \item Multiaxial constitutive behavior of an interstitial-free steel: measurements through X-ray and digital image correlation, \underline{Y. Jeong}, M. Iadicola, T. Gn\"{a}upel-Herold, A. Creuziger (in preparation)
  \item Multi-axial stress analysis using in-situ X-ray diffraction and its uncertainty estimation based on elasto-viscoplastic self-consistent crystal plasticity model, \underline{Y. Jeong}, T. Gn\"{a}upel-Herold, M. Iadicola, A. Creuziger (in preparation)
  \item Forming limit prediction using a self-consistent crystal plasticity framework: a case study for BCC fiber textures, \underline{Y. Jeong}, M.-S. Pham, M. Iadicola, A. Creuziger, T. Foecke (submitted to International Journal of Material Forming)
  \item Effect of martensitic phase transformation on the behavior of 304 austenitic stainless steel under tension, H. Wang, \underline{Y. Jeong}, B. Clausen, Y. Liu, R. J. Mccabe, F. Barlat, C. N. Tom\'{e} (submitted to Materials Science and Engineering A)
  \end{itemize}


  \section{CONFERENCE PROCEEDINGS}
  \begin{itemize}
  \item 2015 Forming limit predictions using a self-consistent crystal plasticity model: a parametric study, Key Engineering Materials, \underline{Y. Jeong}, M.-S. Pham, M. Iadicola, A. Creuziger (ESAFORM)
  \item 2011 Microstructural and crystallographic aspects of yield surface evolution, Materials Science Forum, \underline{Y. Jeong}, F. Barlat, M.-G. Lee
  \item 2011 Crystal Plasticity Predictions of Forward-Reverse Simple Shear Flow Stress, Materials Science Forum, \underline{Y. Jeong}, F. Barlat, M.-G. Lee
  \end{itemize}

  \section{INTERNATIONAL CONFERENCES}
  \begin{itemize}
  \item 2015 TMS 2015, Orlando, VPSC-FLD: forming limit predictions for BCC fibers, \underline{Y. Jeong}, M.-S. Pham, M. Iadicola, A. Creuziger
  \item 2015 ESAFORM, Forming limit diagram predictions using a self-consistent crystal plasticity model: a parametric study, \underline{Y. Jeong}, M.-S. Pham, M. Iadicola, A. Creuziger
  \item 2014 Symposium of Plasticity, A mean field polycrystal plasticity framework for strain-induced martensitic transformation, \underline{Y. Jeong}, C. Tom\'{e}, B. Clausen, H. Wang, M.-G. Lee, F. Barlat
  \item 2014 International Conference on Multiscale Materials Modeling, In-situ Neutron  Measurement and Modeling of Martensitic Phase Transformation, C. Tom\'{e}, \underline{Y. Jeong}, H. Wang, B. Clausen, F. Barlat
  \item 2014 Materials Science \& Technology 2014 Fall meeting, Evaluation of an X-ray stress technique by means of crystal plasticity, \underline{Y. Jeong}, M.-S. Pham, M. Iadicola, A. Creuziger
  \item 2011 NUMISHEET, Biaxial Deformation Behaviour of AZ31 Magnesium Alloy: Crystal-Plasticity-Based Prediction and Experimental Validation, D. Steglich, \underline{Y. Jeong}, M. O. Andar, T. Kuwabara
  \item 2011 TMS Annual Meeting \& Exhibition, Polycrystal Modelling with Experimental Integration: A symposium Honorning Carlos Tom\'e, application of the VPSC model to a commercial 304 austenitic stainless steel, \underline{Y. Jeong}, F. Barlat
  \item 2011 International Conference on the Textures of Materials 16, Crystal plasticity predictions of cyclic simple shear flow stress, \underline{Y. Jeong}, F. Barlat, M.-G. Lee
  \item 2011 International Conference on the Textures of Materials 16, Microstructural and crystallographic aspects of yield surface evolution, \underline{Y. Jeong}, F. Barlat, M.-G. Lee
  \end{itemize}

  \section{DOMESTIC CONFERENCES}
  \begin{itemize}
  \item 2015 NADDRG (North American Deep Drawing Research Group), Evanston IL, Experimental validation for VPSC-FLD and its application for various BCC fiber textures, \underline{Y. Jeong}, M.-S. Pham, M. Iadicola, A. Creuziger.
  \item 2010 KSTP fall meeting, Investigation on the exponent of non-quadratic anisotropic yield surface models by crystal plasticity, \underline{Y. Jeong}, D. Steglich, M.-G. Lee, F. Barlat
  \end{itemize}

  \section{WORKSHOPS}
  \begin{itemize}
  \item 2010 Young Researcher's Asian Workshop on Advanced Forming Technology, Nagoya, Japan, The Iron and Steel Institute of Japan
  \end{itemize}


\end{resume}
\end{document}
